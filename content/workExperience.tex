%---------------------------------------------------------------------------------------
%	WORK EXPERIENCE
%----------------------------------------------------------------------------------------

\cvsection{WORK EXPERIENCE}

\cvevent
{Fev 16 - Jan 20}
{Project Manager/ Software Architecture}
{Research and Development/ Industrial Automation}
{[EIA-ECOD3] - Responsável por definir e gerenciar equipes de software para implementação de customização de ERP e automação industrial utilizando ferramentas opensource}
{\cvlist{
\item Representação e gerenciamento de equipe para implantação do sistema ERP Odoo no Brasil.
\item Implantação e integração de diversos sistemas/ aplicações utilizando estruturas AWS, Linode, Google Cloud e internas. Para clientes como EPLAN, General Motors, Ross Controls.
\item Gerenciamento e desenvolvimento de diversas reformulações e atualizações em linhas como parceiro da Rockwell Automation e Edge group
}}
{\cvlist {
\item GIT (on GitLab) para todos os projetos, Ansible + Docker
\item Frameworks para python como Django,
Flask, Odoo OEE.
\item Frameworks para Javascript com React, Preact.
\item Incluso no desenvolvimento Front-end o uso de styled-components, axios, localstorage, redux com reducers e loggers, router, webpack, eslint com standard ou typescript, linter para commit e até gitmoji.\\
\iconhref {Youtube}{12}{Preact, Redux, Axios, Styled-components and PLC integration}
{https://www.youtube.com/watch?v=4gFQmoYwO88}{black}
\item PostgreSQL, MongoDB
\item Diversas ferramentas e projetos opensource como Zabbix, Axterisk, Matrix, Webdav, Bareos, Traefik, Pfsense
}}
{\cvlist{
\item Automação e IOT para implantação de linha de pintura com tecnologia KTL na Randon Implementos (Caxias do Sul), incluiu elementos como Elipse (SCADA)
sistema supervisor, 4 IHMs, API para OPCUA com NODEJS,
para o consumo de infra-estrutura foi utilizado Grafana na exibição
\iconhref {Github}{12}{openCodeInEnglish}
{https://github.com/fesnavarro/eCoatKtl}{black}
\iconhref {Youtube}{12}{start-up}
{https://www.youtube.com/watch?v=G0fXkKrFBfc&feature=youtu.be}{black}
\item Utilizando ESP32 (com mais recursos internos do que um Arduino) para coleta local e transmissão, em adição a um RapsberryPI como gateway para inserir dados coletados através de API do sistema de ERP local da equipe de manutenção das linhas de solda da General Motors do Brasil em São Caetano do Sul.
}}


\cvevent
{Jan 08 - Fev 16}
{Project Manager / Sofware Enginneer}
{Industrial Automation}
{[EIA-SANMARTIN-GOC] - Working for smaller companies I was responsible for
development and administration at the various levels of these companies
whose main customers were large industries}
{\cvlist{
		\item Technical team management with decision and selection of professionals to compose a
		development and installation team
		\item Support and direct development of commercial proposals,
		training and presentations
	}}
{\cvlist {
		\item Barcode position, encoders, Wifi Radios and bluetooth
		\item Working with PMI methodologies, check lists and quality
		procedures
	}}
{\cvlist{
		\item All idealization of the control architecture, development of
		electrical projects and infrastructure in addition to the
		application of general control of the final chassis assembly line at
		Randon Implementos (Caxias do Sul). The project was mentioned in
		international articles as being a pioneer in applying several technologies
		together.\\
\iconhref {Youtube}{12}{fullProduction} 
{https://youtu.be/_CDIo1rElvk}{black} 
\iconhref {Youtube}{12}{11YearsLaterIPerformedTrainingForOperators} 
{https://www.youtube.com/watch?v=EUxsm1Iy3r0&list=PLz6-G3tr8Wiuq90iuxtEBU1oXETNXDuJE&index=1}{black} 
		\item I coordinated a massive migration from General Motors (SCS) to General Motors (SJC) plants
		in partnership with Tyssen Krupp, in this project I even defined the
		electrical shutdowns and executed the new start-up in the new plant, the
		project was perfect and almost no application changes were necessary,
		except for integrations between the robotic cells and the local network
		\item In response to a call from Eisemman for automation at MAN (Resende),
		I alone made all the necessary changes for the operation of the Carese
		painting line, with more than 200 conveyors, elevators, transfer tables and
		other elements. The main screen had more than 1000 tags interconnected.	
\iconhref {File}{12}{OfflineScreens} 
{https://drive.google.com/open?id=1ImyE51f5dnSwzYilU70aDpS2ChVGb3kR}{black} 
\iconhref {File}{12}{OnlineScreens} 
{https://drive.google.com/open?id=1RlpcjE-vmwsb5kKh0y1ASgjCPCaxtctU}{black}
		\item In partnership with the company LAN (China) I was able to meet the
		needs in the start-up of the final assembly line at Nissan (Resende) with
		excellence, at the beginning of the work there were many Brazilian and
		Chinese programmers interacting with some difficulty in an environment of
		generalized delay. On the third day I had already rewritten the Chinese codes
		for the start of the lines and I took over as the main responsible for the
		start-up directly with the Japanese from Nissan
		\item Development of automation control and start-up application for AMBEV
		factory (Curitiba), it was the first experience with the beverage industry
		and the project included an industrial washer controlled by 8 servants, two
		boxers, accelerator tables and system integrated vision
	}}

\cvevent
{Jan 05 - Dec 07}
{Software Enginneer}
{Industrial Automation}
{[SPI-TECNEL] - Responsible for the direct development of applications and
parameterization of automated lines in addition to official technical assistance}
{\cvlist{
		\item Direct development of complete software applications
		\item Emergency technical assistance to a range of customers such as
		Alergan (pharmacy), Taboca (mining company), Mercedez Bens (final assembly), General Motors (bodyshop), Coral (chemical), Cofap (parts) and others, demand that required maturity and procedures within the best PMI practices
		\item Official representative of Rockwell do Brasil in projects and
		assistance
	}}
{\cvlist {
		\item Various models of PLC (LADDER, Block Instruction, STL), HMI,
		Inverters, Gateways and other components of Rockwell (Logix 5, 500, 5000),
		Siemens (S5, S7, TIA), Mitsubish, Owron, SEW, ATOS, Schneider, Sick,
		Prosoft
		\item Industrial networks and protocols like Modbus (many), Devicenet,
		Ethernet CIP, Profibus, Profinet, Interbus, DH +, OPC and OPCUA
		\item Integration with robots and tools Fanuc, Kuka, Motoman
	}}
{\cvlist{
		\item Automation of a mining line in the Amazon rainforest under social
		isolation for the production of nickel, using several industrial networks
		integrated in the same system as Modbus. All the work done without external
		assistance during 55 days of start up, there was no internet or any external
		communication within the indigenous village
		\item Software design and development at General Motors for the PRISMA
		automation line, 44 robots and 4 integrated contrologix, that was my first big
		project as a software enginneer, the project had a partnership with Daefuku, which provided
		equipment for the aero system and infrastructure for the new RFID system
	}}
